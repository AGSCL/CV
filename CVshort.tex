%!TEX TS-program = xelatex
%!TEX encoding = UTF-8 Unicode
% Awesome CV LaTeX Template for CV/Resume
%
% This template has been downloaded from:
% https://github.com/posquit0/Awesome-CV
%
% Author:
% Claud D. Park <posquit0.bj@gmail.com>
% http://www.posquit0.com
%
%
% Adapted to be an Rmarkdown template by Mitchell O'Hara-Wild
% 23 November 2018
%
% Template license:
% CC BY-SA 4.0 (https://creativecommons.org/licenses/by-sa/4.0/)
%
%-------------------------------------------------------------------------------
% CONFIGURATIONS
%-------------------------------------------------------------------------------
% A4 paper size by default, use 'letterpaper' for US letter
\documentclass[11pt,a4paper,]{awesome-cv}

% Configure page margins with geometry
\usepackage{geometry}
\geometry{left=1.4cm, top=.8cm, right=1.4cm, bottom=1.8cm, footskip=.5cm}


% Specify the location of the included fonts
\fontdir[fonts/]

% Color for highlights
% Awesome Colors: awesome-emerald, awesome-skyblue, awesome-red, awesome-pink, awesome-orange
%                 awesome-nephritis, awesome-concrete, awesome-darknight

\definecolor{awesome}{HTML}{2f40d8}

% Colors for text
% Uncomment if you would like to specify your own color
% \definecolor{darktext}{HTML}{414141}
% \definecolor{text}{HTML}{333333}
% \definecolor{graytext}{HTML}{5D5D5D}
% \definecolor{lighttext}{HTML}{999999}

% Set false if you don't want to highlight section with awesome color
\setbool{acvSectionColorHighlight}{true}

% If you would like to change the social information separator from a pipe (|) to something else
\renewcommand{\acvHeaderSocialSep}{\quad\textbar\quad}

\def\endfirstpage{\newpage}

%-------------------------------------------------------------------------------
%	PERSONAL INFORMATION
%	Comment any of the lines below if they are not required
%-------------------------------------------------------------------------------
% Available options: circle|rectangle,edge/noedge,left/right

\name{Andrés}{González Santa Cruz}

\position{Estudiante Doctorado en Salud Pública UCH}
\address{Santiago de Chile}

\email{\href{mailto:gonzalez.santacruz.andres@gmail.com}{\nolinkurl{gonzalez.santacruz.andres@gmail.com}}}
\orcid{0000-0002-5166-9121}
\github{AGSCL}
\linkedin{andrés-gonzález-santa-cruz}

% \gitlab{gitlab-id}
% \stackoverflow{SO-id}{SO-name}
% \skype{skype-id}
% \reddit{reddit-id}


\usepackage{booktabs}

\providecommand{\tightlist}{%
	\setlength{\itemsep}{0pt}\setlength{\parskip}{0pt}}

%------------------------------------------------------------------------------


\geometry{left=.5cm, top=.75cm, right=.5cm, bottom=.75cm, footskip=0.5cm}
\setlength{\parskip}{0.5em}
\setlength{\baselineskip}{1.2em}

% Pandoc CSL macros

\begin{document}

% Print the header with above personal informations
% Give optional argument to change alignment(C: center, L: left, R: right)
\makecvheader

% Print the footer with 3 arguments(<left>, <center>, <right>)
% Leave any of these blank if they are not needed
% 2019-02-14 Chris Umphlett - add flexibility to the document name in footer, rather than have it be static Curriculum Vitae
\makecvfooter
  {NA}
    {Andrés González Santa Cruz~~~·~~~CV proveniente de
\href{hhttps://github.com/AGSCL/CV}{https://github.com/AGSCL/CV}}
  {\thepage}


%-------------------------------------------------------------------------------
%	CV/RESUME CONTENT
%	Each section is imported separately, open each file in turn to modify content
%------------------------------------------------------------------------------



\section{Experiencia laboral}\label{experiencia-laboral}

\fontsize{9pt}{1em}\color{text}
\begin{cventries}
    \cventry{Diplomado Data Science para Ciencias Sociales, Universidad Diego Portales}{Docencia}{Santiago, Chile}{Noviembre 2024 --> Enero 2025}{}\vspace{-4.0mm}
    \cventry{Carrera Medicina, Universidad de Chile}{Docencia}{Santiago, Chile}{Agosto 2024 --> Diciembre 2024}{}\vspace{-4.0mm}
    \cventry{Núcleo Milenio para la Evaluación y Análisis de Políticas de Drogas (nDP), dirigido por Álvaro Castillo-Carniglia}{Asistente de Investigación}{Santiago, Chile}{Marzo 2022 --> Agosto 2023}{}\vspace{-4.0mm}
    \cventry{Magíster en Métodos de Investigación Social, Universidad Diego Portales}{Docencia}{Santiago, Chile}{Noviembre 2022 --> Noviembre 2024}{}\vspace{-4.0mm}
    \cventry{Magíster en Métodos de Investigación Social, Universidad Diego Portales}{Ayudantía}{Santiago, Chile}{Abril 2022 --> Diciembre 2022}{}\vspace{-4.0mm}
    \cventry{Carrera Data Science, Universidad Mayor}{Docencia}{Santiago, Chile}{Agosto 2021 --> Enero 2022}{}\vspace{-4.0mm}
    \cventry{Proyecto Fondecyt 1191282, a cargo de Álvaro Castillo-Carniglia y Proyecto Fondecyt 1170239, a cargo de Elisa Ansoleaga Moreno}{Asistente de Investigación}{Santiago, Chile}{Marzo 2019 --> Marzo 2022}{}\vspace{-4.0mm}
    \cventry{Departamento de Desarrollo de Personas, Universidad de Santiago de Chile}{Analista de Desarrollo de Personas}{Santiago, Chile}{Enero 2014 --> Diciembre 2019}{}\vspace{-4.0mm}
\end{cventries}

\section{Actividades de
investigación}\label{actividades-de-investigaciuxf3n}

\subsection{Publicaciones}\label{publicaciones}

\fontsize{9.5pt}{1.2em}\color{text}
\begin{cventries}
    \cventry{}{}{}{2024}{\begin{cvitems}
\item Dinamarca-Aravena KA, Rocha-Jiménez T, Morales-Miranda S, Castillo-Carniglia Á, González-Santa Cruz A, Respondent Driving Sampling online (Web RDS) as a strategy to access hard-to-reach but non-hidden populations: the case of health professionals working in Chilean schools. Journal of Social Research Methodology (Manuscrito enviado)
\item Castellano J, González-Santa Cruz A, Castillo-Carniglia Á, Gaete J, Association between living arrangements and time to drop out in patients between 18 and 29 years of age, under treatment for substance abuse disorders in Chile, 2010-2019. Pre-print. 10.21203/rs.3.rs-4276312/v1
\item González-Santa Cruz A, Ansoleaga-Moreno E, Otra lección de la pandemia: relevancia del apoyo social para la protección de la salud mental en trabajadores de la salud. Revista Médica de Chile (Aceptado)
\item Saldías Fernández MA, Gónzalez-Santa-Cruz A, Martínez-Ordenes M, Parra-Giordano D, Factores sociodemográficos-sanitarios sobre la decisión de interrupción del embarazo, Chile: un análisis de clases latentes. Global health promotion. (Aceptado)
\item Mateo-Pinones M, González-Santa Cruz A, Bond C, McGee T, Payne J, \& Castillo-Carniglia Á, Substance use treatment completion and criminal justice system contact in Chile: A retrospective, linked data, cohort study. Addiction. 10.1111/add.16488
\item Bórquez I, Cerdá M, González-Santa Cruz A, Krawczyk N, Castillo-Carniglia Á, Longitudinal trajectories of substance use disorder treatment use: A latent class growth analysis using a national cohort in Chile. Addiction, 119(4), 10.1111/add.16412
\item Saldias-Fernández MA, Gónzalez-Santa-Cruz A, Parra-Giordano D, Interrupción del embarazo en Chile 2018-2020: rol de las variables sociodemográficas y sanitarias sobre la decisión de la mujer. Salud Publica Mex [Internet]. 10.21203/rs.3.rs-4276312/v1
\end{cvitems}}
    \cventry{}{}{}{2023}{\begin{cvitems}
\item Castillo-Carniglia A, González-Santa Cruz A, Mauro PM, Sapag J, Martins SS, Ruiz-Tagle J, Gaete J, Cerdá M, Effect of residential versus ambulatory substance use disorders treatment on readmission risk in a register-based national retrospective cohort. Social Psychiatry and Psychiatric Epidemiology (Manuscrito enviado)
\item Lagos-Barrios R, González-Santa Cruz A, Medina-Marín F, Flores-Alvarado S, Aguero-Jiménez A, Ciencia de Datos para la Salud Pública: Apuntes de una conversación en desarrollo, Inferencias−Boletín de Bioestadística, 8, https://revistasdex.uchile.cl/index.php/int/article/view/12844/12864
\item Lagos-Barrios R, Flores-Alvarado S, González-Santa Cruz A, Medina-Marín F, Ciencia de Datos para la Salud Pública: Una conversación necesaria, Inferencias−Boletín de Bioestadística, 7, https://revistasdex.uchile.cl/index.php/int/article/view/12665/12685
\item Ruiz-Tagle J, González-Santa Cruz, A, Rocha-Jiménez, T, Castillo-Carniglia, A, Does substance use disorder treatment completion reduce the risk of treatment readmission in Chile? Drug and Alcohol Dependence, 248, 109907. 10.1016/j.drugalcdep.2023.109907
\end{cvitems}}
    \cventry{}{}{}{2019-2022}{\begin{cvitems}
\item Mateo-Pinones, M, González-Santa Cruz, A, Portilla-Huidobro, R, Castillo-Carniglia, A, Evidence-based policymaking: Lessons from the Chilean Substance Use Treatment Policy, International Journal of Drug Policy, 109, 103860. 10.1016/j.drugpo.2022.103860
\item González-Santa Cruz A, Ansoleaga-Moreno E, Validación de la Escala de Liderazgo Destructivo y del Cuestionario de Conductas Negativas-Revisado en Chile, Psykhe, 32(2), 10.7764/psykhe.2020.27999.
\item Tapia-Munoz T, González-Santa Cruz A, Clarke H, Morris W, Palmeiro Silva Y, Allel K, COVID 19 attributed mortality and ambient temperature A global ecological study using a two-stage regression model, Pathogens and Global Health, 2007336, 10.1080/20477724.2021.2007336
\item Olivari CF, González-Santa Cruz A, Mauro PM, Martins SS, Sapag J, Gaete J, Cerdá M, Castillo-Carniglia A, Treatment outcome and readmission risk among women in women-only versus mixed-gender drug treatment programs in Chile, Journal of Substance Abuse Treatment, 108616, 10.1016/j.jsat.2021.108616
\item Gajardo AIJ, Wagner TD, Howell K, González-Santa Cruz A, Kaufman JS, Castillo-Carniglia A, Effects of 2019's social protests on emergency health services utilization and case severity in Santiago, Chile, The Lancet Regional Health Americas, 100082, 10.1016/j.lana.2021.100082
\item González-Santa Cruz A, Toro Cifuentes JP, Culturas organizacionales y factores de riesgo psicosociales en organizaciones chilenas Un analisis de clases latentes, Psicoperspectivas, 20, 1, 10.5027/psicoperspectivas-Vol20-Issue1-fulltext-2006
\item Castillo-Carniglia A, González-Santa Cruz A, Cerdá M, Delcher C, Shev A, Wintemute GJ, Henry SG, Changes in opioid prescribing after implementation of mandatory registration and proactive reports within California's prescription drug monitoring program, Drug and alcohol dependence, 1, 221, /j.drugalcdep.2020.108405
\item Ansoleaga E, Ahumada M, González-Santa Cruz A, Association of Workplace Bullying and Workplace Vulnerability in the Psychological Distress of Chilean Workers, International journal of environmental research and public health, 16, 20, 1-14, 10.3390/ijerph16204039
\end{cvitems}}
\end{cventries}

\vspace{0.8cm}

\subsection{Seminarios/congresos}\label{seminarioscongresos}

\fontsize{9.5pt}{1.2em}\color{text}
\begin{cventries}
    \cventry{}{}{Poster}{2024}{\begin{cvitems}
\item González-Santa Cruz A, Castillo-Carniglia Á, Trayectorias de hospitalización por condiciones vinculadas a trastornos de salud mental y consumo de sustancias en usuarios/as jóvenes de población general y pertenecientes a pueblos originarios, 2018-2021, Chile, XVLII Jornadas Nacionales de Estadística, Valdivia
\item González-Santa Cruz A, Castillo-Carniglia Á, Mortality following substance use disorder treatment: population-based record-linkage retrospective cohort design, Annual Meeting of the Society of Epidemiologic Research, Texas, US
\end{cvitems}}
    \cventry{}{}{Poster}{2023}{\begin{cvitems}
\item Mateo-Pinones M, González-Santa Cruz A, Castillo-Carniglia Á, Assessing the impact of substance use treatment for preventing criminal justice system contact in Chile, Annual Meeting of the Society of Epidemiologic Research, Portland, US
\item González-Santa Cruz A, Ruiz-Tagle J, Mateo-Pinones M, Castillo-Carniglia Á, Poly-substance use, treatment completion, and contact with the justice system: a multistate analysis of treatments for substance use disorders between 2010-2019 in Chile
, Portland, US
\item González-Santa Cruz A, Ansoleaga-Moreno E, Plaza de los Reyes M, Relevancia del apoyo social para la protección de la salud mental en el personal sanitario, Congreso Chileno de Salud Pública y Epidemiología, Temuco
\item González-Santa Cruz A, Castellano J, Task development, living conditions and time to drop-out in patients of 18-29 years of age, under treatment for substance abuse disorders in Chile, 2010-2019, 2023 NIDA International Forum, US (virtual)
\end{cvitems}}
    \cventry{}{}{Poster}{2019-2021}{\begin{cvitems}
\item González-Santa Cruz A, Castillo Carniglia A, Ambulatory or residential? a multi-state analysis of treatments for substance use disorders, Chile, College on Problems of Drug Dependence 83rd Annual Scientific Meeting, Nashville, Tennessee, US (online por COVID-19)
\item Olivari CF, González-Santa Cruz A, Castillo Carniglia A, Treatment outcome and readmission risk among women in women only versus mixed gender drug treatment programs in Chile, College on Problems of Drug Dependence 83rd Annual Scientific Meeting, Nashville, Tennessee, US (online por COVID-19)
\item Castillo-Carniglia A, Gajardo AIJ., Wagner TD., Howell K, González-Santa Cruz A, Kaufman JS, Effects of 2019's social protests on emergency health services utilization and case severity in Santiago, Chile, 54rd Annual Meeting of the Society of Epidemiologic Research, San Diego, US (online por COVID-19)
\item Castillo -Carniglia A, González-Santa Cruz A, Cerdá M, Delcher C, Shev A, Wintemute GJ, Henry SG, Changes in opioid prescribing after implementation of mandatory registration and proactive reports within California's prescription drug monitoring program, 53rd Annual Meeting of the Society of Epidemiologic Research, Boston, US (online por COVID-19)
\item González-Santa Cruz A, Toro Cifuentes JP, Percepcion de culturas organizacionales y su relación con conductas de liderazgo destructivo y prevalencia de formas de violencia laboral en una muestra representativa de trabajadores chilenos, Red Iberoamericana de Psicologia de las Organizaciones y del Trabajo RIPOT, Montevideo, Uruguay
\item González-Santa Cruz A, Ansoleaga-Moreno E, Propiedades Psicometricas de Escala de Liderazgo Destructivo y Cuestionario de Conductas Negativas Revisado en una Muestra Multiocupacional de trabajadores dependientes de Gran Santiago, Gran Valparaiso y Gran Concepcion, Chile, I Congreso Iberoamericano de Psicologia de las Organizaciones y el Trabajo, Montevideo, Uruguay
\end{cvitems}}
\end{cventries}

\section{Estudios formales de
interés}\label{estudios-formales-de-interuxe9s}

\fontsize{9.5pt}{1.2em}\color{text}
\begin{cventries}
    \cventry{Doctorado en Salud Pública (c)}{Universidad de Chile}{Santiago, Chile}{2022 --> Presente}{}\vspace{-4.0mm}
    \cventry{Diplomado en Bioestadística Avanzada}{Universidad Católica}{En-línea}{2021 --> 2021}{}\vspace{-4.0mm}
    \cventry{Magíster en Métodos para la Investigación Social}{Universidad Diego Portales}{Santiago, Chile}{2018 --> 2019}{}\vspace{-4.0mm}
    \cventry{Diplomado en Métodos Cuentitativos para la Investigación Social}{Universidad Diego Portales}{Santiago, Chile}{2017 --> 2017}{}\vspace{-4.0mm}
    \cventry{Título profesional de Psicólogo}{Universidad Diego Portales}{Santiago, Chile}{2009 --> 2014}{}\vspace{-4.0mm}
\end{cventries}

\pagebreak

\linebreak



\end{document}
